\documentclass[titlepage,a4paper,12pt,norsk]{IMFeksamen}
\geometry{left=3.5cm,right=3.5cm,bottom=2cm}
\usepackage[utf8]{inputenc}
\trykkinfo[tosidig,sorthvit]
\emnekode{TMA4110}
\emnenavn{Matematikk 3}
\eksamensdato{4. desember 2018}
\eksamenstid{09:00–13:00}
\fagligkontaktinfo{Øystein Skartsæterhagen og Morten Nome}{95 92 55 96}
\hjelpemiddel{Ingen trykte eller håndskrevne hjelpemidler tillatt.
 Bestemt, enkel kalkulator tillatt.
 (Casio fx-82ES PLUS, Casio fx-82EX,
  Citizen SR-270X, Citizen SR-270X College,
  Hewlett Packard HP30S)}
\anneninfo{Eksamenen består av ti oppgaver
Hver av disse teller like mye.
Alle svar må begrunnes.
}
\runninghead{TMA4110 Matematikk 3 -- Eksamen høsten 2018}
\usepackage[T1]{fontenc}
\usepackage{lmodern,amsmath,amssymb,amsfonts}
\usepackage{mathrsfs}
\usepackage{systeme}

\newcommand{\N}{\mathbb{N}}
\newcommand{\Z}{\mathbb{Z}}
\newcommand{\Q}{\mathbb{Q}}
\newcommand{\R}{\mathbb{R}}
\newcommand{\C}{\mathbb{C}}

\newcommand{\M}{\mathcal{M}} % vektorrom av matriser
\newcommand{\Cf}{\mathcal{C}} % vektorrom av kontinuerlige funksjoner
\renewcommand{\P}{\mathcal{P}} % vektorrom av polynomer
\newcommand{\B}{\mathscr{B}} % basis

\renewcommand{\Im}{\operatorname{Im}}
\renewcommand{\Re}{\operatorname{Re}}

\newcommand{\abs}[1]{|#1|}
\newcommand{\intersect}{\cap}
\newcommand{\union}{\cup}
\newcommand{\fcomp}{\circ}
\newcommand{\iso}{\cong}

\newcommand{\roweq}{\sim}
\DeclareMathOperator{\Sp}{Sp}
\DeclareMathOperator{\Null}{Null}
\DeclareMathOperator{\Col}{Col}
\DeclareMathOperator{\Row}{Row}
\DeclareMathOperator{\rank}{rank}
\DeclareMathOperator{\im}{im}
\DeclareMathOperator{\id}{id}
\DeclareMathOperator{\Hom}{Hom}
\newcommand{\tr}{^\top}
\newcommand{\koord}[2]{[\,{#1}\,]_{#2}} % koordinater mhp basis

\newcommand{\V}[1]{\mathbf{#1}}
\newcommand{\vv}[2]{\begin{bmatrix} #1 \\ #2 \end{bmatrix}}
\newcommand{\vvS}[2]{\left[ \begin{smallmatrix} #1 \\ #2 \end{smallmatrix} \right]}
\newcommand{\vvv}[3]{\begin{bmatrix} #1 \\ #2 \\ #3 \end{bmatrix}}
\newcommand{\vvvv}[4]{\begin{bmatrix} #1 \\ #2 \\ #3 \\ #4 \end{bmatrix}}
\newcommand{\vvvvv}[5]{\begin{bmatrix} #1 \\ #2 \\ #3 \\ #4 \\ #5 \end{bmatrix}}
\newcommand{\vn}[2]{\vvvv{#1_1}{#1_2}{\vdots}{#1_#2}}

\newcommand{\e}{\V{e}}
\renewcommand{\u}{\V{u}}
\renewcommand{\v}{\V{v}}
\newcommand{\w}{\V{w}}
\renewcommand{\b}{\V{b}}
\newcommand{\x}{\V{x}}
\newcommand{\0}{\V{0}}

\newenvironment{amatrix}[1]{% "augmented matrix"
  \left[\begin{array}{*{#1}{c}|c}
}{%
  \end{array}\right]
}



\begin{document}

\begin{oppgave}
Finn alle løsninger av følgende likningssystem:
\[
\systeme{
x - y + 2z = 28,
-2x + 5y - 4z = -20,
-x + y - z = -10
}
\]
\end{oppgave}


\begin{oppgave}
Se på følgende vektorer i~$\R^3$:
\[
\v_1 = \vvv{3}{-3}{-6},\quad
\v_2 = \vvv{-2}{2}{4},\quad
\v_3 = \vvv{1}{-1}{8},\quad
\b = \vvv{4}{-9}{3}
\]
Er vektorene $\v_1$, $\v_2$ og~$\v_3$ lineært uavhengige?
Er $\b$ en lineærkombinasjon av $\v_1$, $\v_2$ og~$\v_3$?
% ja; nei
\end{oppgave}


\begin{oppgave}
Finn generell løsning av systemet
\[
\left\{
\begin{aligned}
y_1' &= 7 y_1 - 2 y_2 \\
y_2' &= 2 y_1 + 2 y_2
\end{aligned}
\right.
\]
og skisser fasediagrammet.
\end{oppgave}


\begin{oppgave}
Se på de tre punktene
\[
\vv{0}{-1},\quad
\vv{1}{1}\quad\text{og}\quad
\vv{2}{7}
\]
i $\R^2$.

Finn andregradspolynomet
$p(x) = ax^2 + bx + c$
som går gjennom alle disse punktene.

Bruk minste kvadraters metode til å finne  førstegradspolynomet
$q(x) = dx + e$
som passer best til de tre punktene.

Tegn grafene til $p$ og~$q$.
\end{oppgave}


\begin{oppgave}
La $A$ være følgende matrise:
\[
A =
\begin{bmatrix}
 9 & -3 \\
-3 &  2
\end{bmatrix}
\]
Finn alle $2 \times 2$-matriser~$X$ som er løsninger av likningen
$AX = XA$.
\end{oppgave}


\begin{oppgave}
Finn en ortogonal basis for underrommet av~$\R^4$ utspent av
disse vektorene:
\[
\v_1 = \vvvv{2}{1}{1}{0},\quad
\v_2 = \vvvv{1}{0}{-2}{1},\quad
\v_3 = \vvvv{1}{1}{1}{0},\quad
\v_4 = \vvvv{2}{1}{-1}{1}
\]
\end{oppgave}


\begin{oppgave}
La $R$ være følgende matrise:
\[
R =
\begin{bmatrix}
        1/2 & \sqrt{3}/2 \\
-\sqrt{3}/2 &        1/2
\end{bmatrix}
\]
Regn ut $R^{\,42}$.
\end{oppgave}


\begin{oppgave}
La $A$ være følgende komplekse matrise:
\[
A =
\begin{bmatrix}
2  &  i &  5 - 3i \\
4  & 2i & 10 + 2i \\
2i & -1 &  4 + 6i
\end{bmatrix}
\]
Først: Finn en basis for $\Null A$ og en basis for $\Col A$.

Deretter: Finn alle vektorer $\v$ i~$\C^3$ som er slik at
$A\v = \0$
og
$\| \v \| = 1$.
% basis Null A: (\vvv{i}{-2}{0})
% basis Col A: 
% \v = \vvv{i/\sqrt{5}}{-2/\sqrt{5}}{0} \cdot e^{\theta i} for \theta \in [0,2\pi)
\end{oppgave}


\begin{oppgave}
Husk at vi skriver $\M_2$ for vektorrommet som består av alle reelle
$2 \times 2$-matriser.
Definer en funksjon $T \colon \M_2 \to \M_2$ ved
\[
T(M) = M - M\tr.
\]
Vis at $T$ er en lineærtransformasjon, og finn $\ker T$ og~$\im T$.
\end{oppgave}


\begin{oppgave}
La $A$ være en $m \times n$-matrise med lineært uavhengige kolonner,
og la $B$ og~$C$ være $n \times p$-matriser.
Vis at hvis $AB = AC$, så er $B = C$.
\end{oppgave}


\end{document}
