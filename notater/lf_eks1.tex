\documentclass[notitlepage,a4paper,12pt,norsk]{IMFeksamen}
\usepackage[utf8]{inputenc}
\emnekode{TMA4110}
\emnenavn{Matematikk 3 -- EKSEMPEL~1}
\runninghead{TMA4110 -- Eksamen høsten 2018 -- EKSEMPEL~1 -- Løsning}
\usepackage[T1]{fontenc}
\usepackage{lmodern,amsmath,amssymb,amsfonts}
\usepackage{mathrsfs}
\usepackage{systeme}
\usepackage{tikz}
\usepackage{pgfornament}

\newcommand{\N}{\mathbb{N}}
\newcommand{\Z}{\mathbb{Z}}
\newcommand{\Q}{\mathbb{Q}}
\newcommand{\R}{\mathbb{R}}
\newcommand{\C}{\mathbb{C}}

\newcommand{\M}{\mathcal{M}} % vektorrom av matriser
\newcommand{\Cf}{\mathcal{C}} % vektorrom av kontinuerlige funksjoner
\renewcommand{\P}{\mathcal{P}} % vektorrom av polynomer
\newcommand{\B}{\mathscr{B}} % basis

\renewcommand{\Im}{\operatorname{Im}}
\renewcommand{\Re}{\operatorname{Re}}

\newcommand{\abs}[1]{|#1|}
\newcommand{\intersect}{\cap}
\newcommand{\union}{\cup}
\newcommand{\fcomp}{\circ}
\newcommand{\iso}{\cong}

\newcommand{\roweq}{\sim}
\DeclareMathOperator{\Sp}{Sp}
\DeclareMathOperator{\Null}{Null}
\DeclareMathOperator{\Col}{Col}
\DeclareMathOperator{\Row}{Row}
\DeclareMathOperator{\rank}{rank}
\DeclareMathOperator{\im}{im}
\DeclareMathOperator{\id}{id}
\DeclareMathOperator{\Hom}{Hom}
\newcommand{\tr}{^\top}
\newcommand{\koord}[2]{[\,{#1}\,]_{#2}} % koordinater mhp basis

\newcommand{\V}[1]{\mathbf{#1}}
\newcommand{\vv}[2]{\begin{bmatrix} #1 \\ #2 \end{bmatrix}}
\newcommand{\vvS}[2]{\left[ \begin{smallmatrix} #1 \\ #2 \end{smallmatrix} \right]}
\newcommand{\vvv}[3]{\begin{bmatrix} #1 \\ #2 \\ #3 \end{bmatrix}}
\newcommand{\vvvv}[4]{\begin{bmatrix} #1 \\ #2 \\ #3 \\ #4 \end{bmatrix}}
\newcommand{\vvvvv}[5]{\begin{bmatrix} #1 \\ #2 \\ #3 \\ #4 \\ #5 \end{bmatrix}}
\newcommand{\vn}[2]{\vvvv{#1_1}{#1_2}{\vdots}{#1_#2}}

\newcommand{\e}{\V{e}}
\renewcommand{\u}{\V{u}}
\renewcommand{\v}{\V{v}}
\newcommand{\w}{\V{w}}
\renewcommand{\a}{\V{a}}
\renewcommand{\b}{\V{b}}
\renewcommand{\c}{\V{c}}
\newcommand{\x}{\V{x}}
\newcommand{\0}{\V{0}}

\newenvironment{amatrix}[1]{% "augmented matrix"
  \left[\begin{array}{*{#1}{c}|c}
}{%
  \end{array}\right]
}

\newcommand{\oppgslutt}{
\begin{center}
\pgfornament[width=6cm]{88}
\end{center}
}
\newenvironment{losning}{\begin{oppgave}}{\oppgslutt\end{oppgave}}


\begin{document}

\begin{center}
\textbf{\large Løsningsforslag} \\[3pt]
\pgfornament[width=6cm]{88}
\end{center}
\vspace{-10pt}


\begin{losning}
Vi setter opp totalmatrisen og gausseliminerer:
\begin{align*}
\begin{amatrix}{4}
0 & 2 & 4 &   2 & 6 \\
2 & 1 & 2 & -13 & 3 \\
1 & 3 & 6 &  -4 & 9
\end{amatrix}
&\roweq 
\begin{amatrix}{4}
1 & 3 & 6 &  -4 & 9 \\
2 & 1 & 2 & -13 & 3 \\
0 & 2 & 4 &   2 & 6
\end{amatrix}
\roweq
\begin{amatrix}{4}
1 & 3 & 6 &  -4 & 9 \\
0 & -5 & -10 & -5 & -15 \\
0 & 2 & 4 &   2 & 6
\end{amatrix}\\
&\roweq 
\begin{amatrix}{4}
1 & 3 & 6 &  -4 & 9 \\
0 & 1 & 2 & 1 & 3 \\
0 & 2 & 4 &   2 & 6
\end{amatrix}
\roweq 
\begin{amatrix}{4}
1 & 3 & 6 &  -4 & 9 \\
0 & 1 & 2 & 1 & 3 \\
0 & 0 & 0 &  0 & 0
\end{amatrix}
\\
&\roweq
\begin{amatrix}{4}
1 & 0 & 0 &  -7 & 0 \\
0 & 1 & 2 & 1 & 3 \\
0 & 0 & 0 &  0 & 0
\end{amatrix}.
\end{align*}
Dette gir likningene 
\[
x_1-7x_4=0
\]
og
\[
x_2+2x_3+x_4=3.
\]
Vi har altså to frie variabler;
det finnes mange valg av parametrisering.
Du kan for eksemepl velge $x_3=s$ og $x_4=t$ som parametere.
Dette gir løsningene
\[
% x1=7t
% x2=3-2s-t
% x3=s
% x4=t
\x = \vvvv{0}{3}{0}{0} + \vvvv{0}{-2}{1}{0} s + \vvvv{7}{-1}{0}{1} t
\qquad
\text{for alle reelle tall $s$ og~$t$.}
\]
\end{losning}


\begin{losning}
Den ortogonale projeksjonen av $\u$ på~$\v$ er:
\raisebox{-11em}[0pt][0pt]{
\hbox{}
\hspace{5em}
\begin{tikzpicture}[scale=1]
\draw[->] (-1.5,0) -- (2.5,0);
\draw[->] (0,-1.5) -- (0,3.5);
\foreach \x in {-1,1,2}
\draw (\x,.1) -- (\x,-.1);
\foreach \y in {-1,1,2,3}
\draw (.1,\y) -- (-.1,\y);
\draw[->] (0,0) -- (1,3) node[anchor=west] {$\u$};
\draw[->] (0,0) -- (1,-1) node[anchor=west] {$\v$};
\draw[->] (0,0) -- (-1,1) node[anchor=south] {$\u_1$};
\draw[->] (0,0) -- (2,2) node[anchor=west] {$\u_2$};
\end{tikzpicture}
}
\[
\u_1
 = \frac{\v\tr \u}{\v\tr \v} \v
 = \frac{-2}{2} \vv{1}{-1}
 = \vv{-1}{1}
\]
Vi skal ha $\u = \u_1 + \u_2$.  Da får vi:
\[
\u_2 = \u - \u_1 = \vv{1}{3} - \vv{-1}{1} = \vv{2}{2}
\]
\end{losning}


\begin{losning}
La
\[
A=
\begin{bmatrix}
4 & 0 & -16 \\
0 & 2 & 0 \\
-3 & 0 & 6
\end{bmatrix}
\]
være koeffisientmatrisen.  Vi begynner med å finne egenverdiene og
egenvektorene til~$A$.  Det karakteristiske polynomet er:
\begin{align*}
\det (A - \lambda I_3)
&=
\begin{vmatrix}
4-\lambda & 0 & -16 \\
0 & 2-\lambda & 0 \\
-3 & 0 & 6-\lambda
\end{vmatrix}
=
(2-\lambda)
\begin{vmatrix}
4-\lambda & -16 \\
-3 &  6-\lambda
\end{vmatrix}
= (2 - \lambda) (\lambda^2 - 10\lambda - 24)
\end{align*}
Egenverdiene er altså løsningene av likningen
\[
(2 - \lambda) (\lambda^2 - 10\lambda - 24) = 0.
\]
% Vi ser umiddelbart at $\lambda_1=2$ er en egenverdi.
% Bruk abc-formelen for å finne resten:
% \[
% \lambda = \frac{10 \pm \sqrt{(-10)^2-4(-24)}}{2}=5\pm 7.
% \]
% Dette gir tre egenverdier: $2$, $-2$ og~$12$.
Det betyr at det er tre egenverdier: $2$, $-2$ og~$12$.

Vi finner en egenvektor som hører til egenverdien~$2$
ved å løse likningen $(A - 2 I_3) \x = \0$:
\begin{align*}
A - 2 I_3 &=
\begin{bmatrix}
2 & 0 & -16 \\
0 & 0 & 0 \\
-3 & 0 & 4
\end{bmatrix}
\roweq
\begin{bmatrix}
1 & 0 & -8 \\
-3 & 0 & 4 \\
0 & 0 & 0
\end{bmatrix}
\roweq
\begin{bmatrix}
1 & 0 & -8 \\
0 & 0 & -20 \\
0 & 0 & 0
\end{bmatrix}
\roweq
\begin{bmatrix}
1 & 0 & 0 \\
0 & 0 & 1 \\
0 & 0 & 0
\end{bmatrix}
\end{align*}
Løsningene er
\[
\x = \vvv{0}{1}{0} t
\qquad\text{for alle relle tall~$t$,}
\]
så vi kan for eksempel velge vektoren $(0,1,0)$.

Tilsvarende fremgangsmåte for $-2$ og~$12$ gir henholdsvis
$\vvv{8}{0}{3}$ og $\vvv{-2}{0}{1}$.

Til slutt setter vi inn i formelen for generell løsning:
$$\V{y} = c_1\vvv{0}{1}{0}e^{2t}+c_2\vvv{8}{0}{3}e^{-2t}+c_3 \vvv{-2}{0}{1}e^{12t}.$$
\end{losning}


\begin{losning}
Husk at determinanten ikke endrer seg dersom man legger til et
multiplum av en rad til en annen, og determinanten multipliseres med
$-1$ dersom man bytter om på to rader:
\begin{align*}
\begin{vmatrix}
3  &  1 - i  & i      & 4      \\
3  &  1      & 1 - 2i & 4 + 7i \\
6i &  2 + 2i & -2     & 3i     \\
-3 & -1 + i  & 1      & 3 - 4i
\end{vmatrix}
% &=
% \begin{vmatrix}
% 3  &  1 - i   & i      & 4  \\
% 0  &  i       & 1 - 3i & 7i \\
% 6i &  2 + 2i  & -2     & 3i \\
% -3 & -1 + i   & 1      & 3 - 4i
% \end{vmatrix}
% \\
% &=
% \begin{vmatrix}
% 3  &  1 - i  & i      & 4  \\
% 0  &  i      & 1 - 3i & 7i \\
% 0  &  0      & 0     & -5i \\
% -3 & -1 + i  & 1     & 3 - 4i
% \end{vmatrix}
% \\
&=
\begin{vmatrix}
3 &  1 - i  & i      & 4  \\
0 &  i      & 1 - 3i & 7i \\
0 &  0      & 0      & -5i\\
0 &  0      & 1+i    & 7 - 4i
\end{vmatrix}
\\
&=
-\begin{vmatrix}
3 &  1 - i  & i      & 4     \\
0 &  i      & 1 - 3i & 7i    \\
0 &  0      & 1+i    & 7 - 4i\\
0 &  0      & 0      & -5i
\end{vmatrix}
\\
&= - 3 \cdot i \cdot (1+i) \cdot (-5i)
 = -15-15i
\end{align*}
% Nå kan vi bruke at determinanten til en triangulær matrise er
% produktet av elementene langs diagonalen:
% \[
% \begin{vmatrix}
% 3  &  1 - i  & i      & 4      \\
% 3  &  1      & 1 - 2i & 4 + 7i \\
% 6i &  2 + 2i & -2     & 3i     \\
% -3 & -1 + i  & 1      & 3 - 4i
% \end{vmatrix} 
% \]
\end{losning}


\begin{losning}
Informasjonen i oppgaven er essensielt en diagonalisering av matrisen.
Vi har tre lineært uavhengige egenvektorer
\[
\vvv{1}{2}{-3},\quad
\vvv{1}{3}{-3}\quad
\text{og}\quad
\vvv{-2}{-4}{8}
\]
med tilhørende egenverdier $3$, $3$ og~$-2$.  Vi lager en matrise
\[
V =
\begin{bmatrix}
 1 &  1 & -2 \\
 2 &  3 & -4 \\
-3 & -3 &  8
\end{bmatrix}
\]
med egenvektorene som kolonner, og en diagonalmatrise
\[
D =
\begin{bmatrix}
3 & 0 &  0 \\
0 & 3 &  0 \\
0 & 0 & -2
\end{bmatrix}
\]
med egenverdiene på diagonalen.  Matrisen vi vil finne er da gitt ved
diagonaliseringen $VDV^{-1}$.

For å kunne regne ut matrisen gjenstår det å finne inversen til~$V$.
Vi regner ut denne på vanlig måte:
\begin{align*}
\left[\begin{array}{ccc|ccc}
1 & 1 & -2 & 1 & 0 & 0 \\
2 & 3 & -4 & 0 & 1 & 0 \\
-3 & -3 & 8 & 0 & 0 & 1
\end{array}\right]
&
\roweq
\left[\begin{array}{ccc|ccc}
1 & 1 & -2 & 1 & 0 & 0 \\
0 & 1 & 0 & -2 & 1 & 0 \\
0 & 0 & 2 & 3 & 0 & 1
\end{array}\right]
\roweq
\left[\begin{array}{ccc|ccc}
1 & 1 & 0 & 4 & 0 & 1 \\
0 & 1 & 0 & -2 & 1 & 0 \\
0 & 0 & 2 & 3 & 0 & 1
\end{array}\right]
\\
&
\roweq
\left[\begin{array}{ccc|ccc}
1 & 0 & 0 & 6 & -1 & 1 \\
0 & 1 & 0 & -2 & 1 & 0 \\
0 & 0 & 2 & 3 & 0 & 1
\end{array}\right]
\roweq
\left[\begin{array}{ccc|ccc}
1 & 0 & 0 & 6 & -1 & 1 \\
0 & 1 & 0 & -2 & 1 & 0 \\
0 & 0 & 1 & \frac{3}{2} & 0 & \frac{1}{2}
\end{array}\right]
\end{align*}
Vi leser av at inversen til $V$ er:
\[
V^{-1} =
\begin{bmatrix}
6 & -1 & 1 \\
-2 & 1 & 0 \\
\frac{3}{2} & 0 & \frac{1}{2}
\end{bmatrix}
\]
Nå kan vi regne ut matrisen vi skulle finne:
\[
VDV^{-1} =
\begin{bmatrix}
 1 &  1 & -2 \\
 2 &  3 & -4 \\
-3 & -3 &  8
\end{bmatrix}
\begin{bmatrix}
3 & 0 &  0 \\
0 & 3 &  0 \\
0 & 0 & -2
\end{bmatrix}
\begin{bmatrix}
6 & -1 & 1 \\
-2 & 1 & 0 \\
\frac{3}{2} & 0 & \frac{1}{2}
\end{bmatrix}
=
\begin{bmatrix}
 18 & 0 &   5 \\
 30 & 3 &  10 \\
-60 & 0 & -17
\end{bmatrix}
\]
\end{losning}


\begin{losning}
Vi begynner med å gausseliminere matrisen:
\begin{align*}
\begin{bmatrix}
1 & 5 & 5 \\
2 & 8 & 6 \\
-1 & 3 & 11
\end{bmatrix}
&\roweq
\begin{bmatrix}
1 & 5 & 5 \\
0 & -2 & -4 \\
0 & 8 & 16
\end{bmatrix}
\roweq
\begin{bmatrix}
1 & 5 & 5 \\
0 & 1 & 2 \\
0 & 0 & 0
\end{bmatrix}
\end{align*}
Dette viser at vi har pivotelementer i første og andre kolonne.
Det betyr at
\[
\left( \vvv{1}{2}{-1}, \ \vvv{5}{8}{3} \right)
\]
er en basis for kolonnerommet.
Men denne basisen er ikke ortogonal
(indreproduktet av de to vektorene er~$18$, ikke~$0$).
Vi ortogonaliserer den med Gram--Schmidt-metoden:
\begin{align*}
\v_1 &= \vvv{1}{2}{-1} &
\u_1 &= \v_1 = \vvv{1}{2}{-1} \\
\v_2 &= \vvv{5}{8}{3} &
\u_2 &= \v_2 - \frac{\u_1\tr \v_2}{\u_1\tr \u_1} \u_1
      = \vvv{5}{8}{3} - \frac{18}{6} \vvv{1}{2}{-1}
      = \vvv{2}{2}{6}
\end{align*}
Nå har vi funnet en ortogonal basis $(\u_1, \u_2)$
for kolonnerommet til matrisen.
\end{losning}


\begin{losning}
Skriv $B$ på elementform:
\[
B=\begin{bmatrix}
b_1 & b_4\\
b_2 & b_5\\
b_3 & b_6
\end{bmatrix}
\]
Produktet $AB$ kan nå uttrykkes ved 
\[
AB=
\begin{bmatrix}
3 & 0 & 4 \\
2 & 1 & 1
\end{bmatrix}
\begin{bmatrix}
b_1 & b_4\\
b_2 & b_5\\
b_3 & b_6
\end{bmatrix}=
\begin{bmatrix}
3b_1+4b_3 & 3b_4+4b_6\\
2b_1+b_2+b_3 & 2b_4+b_5+b_6
\end{bmatrix}
\]
Vi ønsker å bestemme $b_i$-ene slik at $AB=I$:
\[
\begin{bmatrix}
3b_1+4b_3 & 3b_4+4b_6\\
2b_1+b_2+b_3 & 2b_4+b_5+b_6
\end{bmatrix}
=
\begin{bmatrix}
1 & 0\\
0 & 1
\end{bmatrix}
\]
Dette er fire likninger med seks ukjente:
\begin{align*}
3b_1+4b_3&=1\\
2b_1+b_2+b_3&=0\\
3b_4+4b_6&=0\\
2b_4+b_5+b_6&=1
\end{align*}
%Totalmatrisen er
%\[
%\begin{bmatrix}
%3 & 0 & 4 & 0 & 0 & 0 & 1\\
%2 & 1 & 1 & 0 & 0 & 0 & 0\\
%0 & 0 & 0 & 3 & 0 & 4 & 0\\
%0 & 0 & 0 & 2 & 1 & 1 & 1
%\end{bmatrix}.
%\]
Vi ser at $b_3=t$ og~$b_6=s$ er frie variabler. Multipliser andre og
fjerde likning med $3$, og bruk første likning for å eliminere $b_1$ i
andre likning; tredje likning for å eliminere $b_4$ i fjerde likning:
\[
b_1 = \frac{1-4t}{3} \qquad\quad
b_2 = \frac{5t-2}{3} \qquad\quad
b_4 = -\frac{4}{3}s \qquad\quad
b_5 = \frac{5s+3}{3}
\]
(Du kan alternativt skrive opp totalmatrisen til systemet og
radredusere.)

Alle løsninger kan dermed parametriseres som
%(du trenger bare å finne en av disse for å svare på oppgaven)
\[
B =
\begin{bmatrix}
(1-4t)/3 & -4s/3 \\
(5t-2)/3 & (5s+3)/3\\
t & s
\end{bmatrix}
\qquad\text{der $s$ og~$t$ er reelle tall.}
\]
Men oppgaven spør bare etter én løsning.
Vi kan for eksempel ta $t$ og~$s$ lik null for og få følgende løsning:
\[
B=
\begin{bmatrix}
\frac{1}{3} & 0\\
-\frac{2}{3} & 1\\
0 & 0
\end{bmatrix}.
\]
\end{losning}


\begin{losning}
Vi vet at $T(\u)$, $T(\v)$ og~$T(\w)$ er lineært uavhengige, hvor $T$ er en lineærtransformasjon. 

Vi ønsker å vise at $\u$, $\v$ og~$\w$ er lineært uavhengige. Anta at de er lineært avhengige (for å vise at dette umulig kan stemme). Dette betyr -- per definisjon -- at det eksisterer koeffisienter $a$, $b$ og~$c$, ikke alle lik null, slik at
\[
a\u+b\v+c\w=\0. 
\]
Anvend $T$ på begge sider for å få 
\[
T(a\u+b\v+c\w)=T(\0). 
\]
Vi vet at $T$ er lineær som gir
\[
aT(\u)+bT(\v)+cT(\w)=\0. 
\]
Denne likningen betyr i så fall at $T(\u)$, $T(\v)$ og~$T(\w)$ også er lineært avhengige. Men vi \emph{vet} jo at de er lineært uavhengige -- antagelsen er altså umulig . Dermed må $\u$, $\v$ og~$\w$ være lineært uavhengige -- som er det vi ønsket å vise.
\end{losning}


\begin{losning}
Vi husker at vi har følgende tre kriterier for å sjekke om en gitt
delmengde $U$ av et vektorrom er et underrom:
\begin{enumerate}
\item Nullvektoren er med i~$U$.
\item For alle vektorer $\u$ og~$v$ i~$U$
er også summen $\u + \v$ med i~$U$.
\item For alle vektorer $\u$ i~$U$ og alle skalarer~$c$
er også produktet $c\u$ med i~$U$.
\end{enumerate}
Vi sjekker at disse tre kriteriene holder når $U$ er mengden av
symmetriske $2 \times 2$-matriser:
\begin{enumerate}
\item Nullvektoren -- altså matrisen
$
\left[
\begin{smallmatrix}
0 & 0 \\
0 & 0
\end{smallmatrix}
\right]
$ --
er symmetrisk, så den er med i~$U$.
\item Hvis
\[
M =
\begin{bmatrix}
a & b \\
b & d
\end{bmatrix}
\quad\text{og}\quad
N =
\begin{bmatrix}
a' & b' \\
b' & d'
\end{bmatrix}
\]
er symmetriske matriser, så er også summen
\[
M + N =
\begin{bmatrix}
a+a' & b+b' \\
b+b' & d+d'
\end{bmatrix}
\]
en symmetrisk matrise.
\item Hvis
\[
M =
\begin{bmatrix}
a & b \\
b & d
\end{bmatrix}
\]
er en symmetrisk matrise og $c$ en skalar, så er også produktet
\[
cM =
\begin{bmatrix}
ca & cb \\
cb & cd
\end{bmatrix}
\]
en symmetrisk matrise.
\end{enumerate}
Dette vil si at $U$ er et underrom av~$\M_2$.

Videre skal vi finne en basis for~$U$.
Vi ser at de tre matrisene
\[
B_1 =
\begin{bmatrix}
1 & 0 \\
0 & 0
\end{bmatrix},\quad
B_2 =
\begin{bmatrix}
0 & 1 \\
1 & 0
\end{bmatrix}\quad
\text{og}\quad
B_3 =
\begin{bmatrix}
0 & 0 \\
0 & 1
\end{bmatrix}
\]
utspenner~$U$, siden enhver symmetrisk matrise
$
M =
\left[
\begin{smallmatrix}
a & b \\
b & d
\end{smallmatrix}
\right]
$
kan skrives som en lineærkombinasjon
\[
M = a B_1 + b B_2 + d B_3
\]
av disse.
Matrisene $B_1$, $B_2$ og~$B_3$ er dessuten lineært uavhengige,
siden likningen
\[
c_1 B_1 + c_2 B_2 + c_3 B_3 =
\begin{bmatrix}
0 & 0 \\
0 & 0
\end{bmatrix}
\]
ikke har andre løsninger enn den trivielle løsningen $c_1=c_2=c_3=0$.
Dette vil si at $(B_1,B_2,B_3)$ er en basis for~$U$.

Basisen vi fant for~$U$ består av tre basisvektorer,
så dimensjonen til $U$ er~$3$.
Vi kan lett sjekke at
\[
\left(
\begin{bmatrix}
1 & 0 \\
0 & 0
\end{bmatrix},\ %
\begin{bmatrix}
0 & 1 \\
0 & 0
\end{bmatrix},\ %
\begin{bmatrix}
0 & 0 \\
1 & 0
\end{bmatrix},\ %
\begin{bmatrix}
0 & 0 \\
0 & 1
\end{bmatrix}
\right)
\]
er en basis for~$\M_2$,
og det betyr at dimensjonen til~$\M_2$ er~$4$.
\end{losning}


\begin{losning}
Rangen til~$A$ er det samme som dimensjonen til kolonnerommet:
\[
r = \dim \Col A
\]
Dette betyr at kolonnerommet har en basis som består av $r$ vektorer.
La $(\b_1, \b_2, \ldots, \b_r)$ være en slik basis,
og la
\[
B =
\begin{bmatrix}
\,
\b_1 & \b_2 & \cdots & \b_r
\,
\end{bmatrix}
\]
være $m \times r$-matrisen med basisvektorene som kolonner.

La $\a_1$, $\a_2$, \ldots, $\a_n$ være kolonnene i~$A$.
Hver kolonne $\a_i$ ligger i kolonnerommet til~$A$,
så den kan skrives som en lineærkombinasjon
\[
\a_i = c_{i1} \b_1 + c_{i2} \b_2 + \cdots + c_{ir} \b_r
\]
av basisvektorene.
Koeffisientene i dette uttrykket kan vi sette sammen til en vektor
\[
\c_i = \vvvv{c_{i1}}{c_{i2}}{\vdots}{c_{ir}}
\]
i~$\R^r$.
Da har vi:
\[
\a_i
= c_{i1} \b_1 + c_{i2} \b_2 + \cdots + c_{ir} \b_r
=
\begin{bmatrix}
\,
\b_1 & \b_2 & \cdots & \b_r
\,
\end{bmatrix}
\vvvv{c_{i1}}{c_{i2}}{\vdots}{c_{ir}}
= B \c_i
\]
Nå har vi definert en vektor $\c_i$ for hver kolonne $\a_i$ fra~$A$;
det betyr at vi har definert $n$ vektorer
$\c_1$, $\c_2$, \ldots, $\c_n$.  Vi lager en $r \times n$-matrise
\[
C =
\begin{bmatrix}
\,
\c_1 & \c_2 & \cdots & \c_n
\,
\end{bmatrix}
\]
med disse vektorene som kolonner.

Nå har vi funnet en $m \times r$-matrise $B$
og en $r \times n$-matrise $C$, og vi har
\[
BC =
\begin{bmatrix}
\,
B \c_1 & B \c_2 & \cdots & B \c_n
\,
\end{bmatrix}
=
\begin{bmatrix}
\,
\a_1 & \a_2 & \cdots & \a_n
\,
\end{bmatrix}
= A.
\]
\end{losning}


\end{document}
