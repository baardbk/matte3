% -*- TeX-master: "oving01"; -*-
\oppgaver{1}

\begin{oppgave}
Hvilke av disse likningene er lineære?
\begin{punkt}
$14x + 3y = 2x + 1 - 5z$
\end{punkt}
\begin{punkt}
$x + 2xy + y = 1$
\end{punkt}
\begin{punkt}
$\frac{x + y}{2} = z$
\end{punkt}
\end{oppgave}

\begin{losning}
Likning (a) og (c) er lineære; (b) er ikke.
\end{losning}


\begin{oppgave}
Lag et lineært likningssystem med to likninger og to ukjente som
\begin{punkt}
\ldots\ har entydig løsning.
\end{punkt}
\begin{punkt}
\ldots\ ikke har noen løsning.
\end{punkt}
\begin{punkt}
\ldots\ har uendelig mange løsninger.
\end{punkt}
\smallskip\noindent
I hver deloppgave: Tegn grafene til de to likningene i systemet ditt.
\end{oppgave}

\begin{losning}
Det finnes mange eksempler som alle tilfredstiller at
\begin{punkt}
\ldots\ linjene skjærer hverandre i ett punkt:
\end{punkt}
\begin{center}
	\begin{tikzpicture}
	\draw[->] (-2,0) -- (2,0) node[right] {$x$};
	\draw[->] (0,-2) -- (0,2) node[above] {$y$};
	\draw[scale=0.7,domain=-2:2,smooth,variable=\x] plot ({\x},{\x});
	\draw[scale=0.7,domain=-2:2,smooth,variable=\y]  plot ({\y},{(1)});
	\end{tikzpicture}
\end{center}
\begin{punkt}
\ldots\ linjene er parallelle:
\end{punkt}
\begin{center}
	\begin{tikzpicture}
	\draw[->] (-2,0) -- (2,0) node[right] {$x$};
	\draw[->] (0,-2) -- (0,2) node[above] {$y$};
	\draw[scale=0.7,domain=-2:2,smooth,variable=\x] plot ({\x},{\x});
	\draw[scale=0.7,domain=-2:2,smooth,variable=\y]  plot ({\y},{(\y+1)});
	\end{tikzpicture}
\end{center}
\begin{punkt}
\ldots\ linjene er helt like:
\end{punkt}
\begin{center}
	\begin{tikzpicture}
	\draw[->] (-2,0) -- (2,0) node[right] {$x$};
	\draw[->] (0,-2) -- (0,2) node[above] {$y$};
	\draw[scale=0.7,domain=-2:2,smooth,variable=\x] plot ({\x},{\x});
	\draw[scale=0.7,domain=-2:2,smooth,variable=\y]  plot ({\y},{(\y)});
	\end{tikzpicture}
\end{center}
\end{losning}



\begin{oppgave}
En lineær likning med to ukjente kan tegnes som en rett linje i $x$--$y$-planet.
\begin{punkt}
Hvordan kan vi på tilsvarende måte se for oss en lineær likning med tre ukjente?
\end{punkt}
\begin{punkt}
Se på følgende likningssystem:
\[
\systeme{
  x + y + z = 5,
  z = 3
}
\]
Tegn en figur som illustrerer løsningene av hver av disse likningene
og løsningene av systemet.
\end{punkt}
\end{oppgave}

\begin{losning}

\begin{punkt}
En lineær likning med tre ukjente kan tegnes som ett plan i $x$--$y$--$z$-rommet.
\end{punkt}
\begin{punkt}
Likningen $z=3$ svarer - geometrisk - til et plan som skjærer $z$-aksen normalt i $z=3$. Vi gjenkjenner også $$x+y+z=5$$ som et plan i rommet. Begge likningene er oppfylt når planene skjærer hverandre. Ved å sette inn $z=3$ i $$x+y+3=5$$ ser vi at $x+y=2$. Løsningene ligger altså på linjen $x+y=2$ i planet $z=3$. Skisse av linjen sett ovenfra:
\begin{tikzpicture}
\draw[->] (-1,0) -- (3,0) node[right] {$x$};
\draw[->] (0,-1) -- (0,3) node[above] {$y$};
\draw[scale=0.7,domain=-1:3,smooth,variable=\x] plot ({\x},{2-\x});
%\draw[scale=0.7,domain=-2:2,smooth,variable=\y]  plot ({\y},{(\y)});
\end{tikzpicture}
\newline
Prøv å skissere dette i $x$--$y$--$z$-rommet.
\end{punkt}
\end{losning}
